% !TEX root = ../main.tex
%===============================================================================
% B04_References.tex
%
% Author: Robert Gaugl
%
% Content: References
%===============================================================================

\section{References}
\label{sec:references}

Each heading must be followed by at least a short paragraph of text. There should be no two consecutive headings without text in between.


%===============================================================================
\subsection{Examples for References}
\label{sec:examplesForReferences}

Many LaTeX editors offer built-in features for creating references using BibTeX or BibLaTeX. The recommended citation style is the IEEE style, which is preset. If a different citation style is desired, this can be changed in the main.tex file.

In Overleaf, references are typically managed by editing the \texttt{.bib} file directly, which is usually named \texttt{Literatur.bib}\footnote{This file can be opened and edited directly within Overleaf via the file tree on the left.}. To cite a source in your document, use the \verb|\cite{}| command with the citation key of the relevant entry.

This is an example of a citation~\cite{Hunt2019}.

If you're using a reference management tool like \href{https://www.zotero.org}{Zotero} (recommended) or Mendeley, you can link those reference management tools with Overleaf. In that case, the manual entry described above becomes unnecessary. This is recommended for larger works like Master's theses. You can find an instruction on how to link Zotero with Overleaf on this website: \href{https://www.overleaf.com/learn/how-to/How_to_link_Zotero_to_your_Overleaf_account}{How to link Zotero to your Overleaf account}.

When citing, a distinction is made between direct and indirect (paraphrased) quotations.


%===============================================================================
\subsubsection{Direct Quotation}
\label{sec:directQuotation}

A direct quote reproduces the content word for word. It should be enclosed in quotation marks and italicized. Such quotes should only be used in rare exceptions.

Example of a direct quote:

\begin{center}
		{\glqq}\textit{Two things are infinite: the universe and human stupidity; and I'm not sure about the universe.}{\grqq} \cite{Einstein1941}
\end{center}


%===============================================================================
\subsubsection{Indirect Quotation}
\label{sec:indirectQuotation}

By default, indirect quotations should be used in academic writing. This means reading the sources and then expressing the content in your \textbf{own words}.

Even if not quoting verbatim, a reference must still be provided so that the origin of the information is traceable.

If the reference refers only to the \textbf{preceding sentence}, it is placed \textbf{before the punctuation mark}. Example:

The European Commission has set the following targets for 2020: 20\% fewer greenhouse gases, 20\% renewable energy, and 20\% better energy efficiency~\cite{EC2008}.

However, if an \textbf{entire paragraph} is based on a single source, the citation is placed \textbf{after the punctuation mark of the last sentence} in the paragraph. For example:

This chapter describes how light or solar radiation is converted into electrical energy within a solar cell. It explains the difference between the external and internal photoelectric effects and why only the internal photoelectric effect is relevant in solar cells. The photovoltaic effect is also explained. Both the internal photoelectric effect and the photovoltaic effect are essential for explaining how a solar cell works. In order for current to flow, free charge carriers are required, which are generated by the internal photoelectric effect. To prevent these free electrons from recombining with a hole immediately, the photovoltaic effect of a pn-junction is used.~\cite{Finke2012}

\textbf{Each paragraph} must have a \textbf{reference} if it contains information from another source. Even if \textbf{multiple paragraphs} are based on the \textbf{same source}, \textbf{each paragraph} must include its own \textbf{reference}.

If an \textbf{entire chapter} is based on a \textbf{single source} (this should only be done in exceptional cases), it can be stated as follows:

Unless otherwise indicated, the following chapter is based on the book {\glqq}Grid-Connected Photovoltaic Systems{\grqq} by Jürgen Schlabbach~\cite{Schlabbach2011}.
