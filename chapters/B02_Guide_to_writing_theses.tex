% !TEX root = ../main.tex
%===============================================================================
% B02_Guide_to_writing_theses.tex
%
% Author: TU Graz
%
% Content: Lorem Ipsum
%===============================================================================

\section{Guide to Writing Theses}
\label{sec:guideToWritingTheses}


%===============================================================================
\subsection{Procedure for Topic Development}
\label{sec:procedureForTopicDevelopment}

In order to successfully complete a thesis within the planned time frame, certain basic rules should be followed that encourage efficient work. The following general procedure is recommended.

\begin{itemize}
    \item At the beginning of the work: Create an outline (table of contents). This provides a basic structure for your work from the very beginning, which aids in structured working. Discuss the outline of the thesis early with your supervisor.
    \item Initially populate the text according to this outline with keywords, then progressively with text, tables, and figures. Keep the table of contents updated at all times.
    \item Collect documents such as literature references, drafts of tables and figures. Make sure to note the sources to save time on later research.
    \item Anticipate the expected results and first formulate the abstract and conclusions (discussion with the supervisor).
    \item Read the completed thesis for spelling errors, readability, style, and logical structure. If possible, leave the work for a few days before reading it again, as blindness to one's own work can develop quickly! For master's theses, have an uninvolved person and/or a colleague review it.
    \item Do not expect grammatical correction from the supervisor!
\end{itemize}


%===============================================================================
\subsection{Literature Research}
\label{sec:literatureResearch}

The starting point of a scientific thesis is usually a literature review to acquire targeted knowledge on the chosen topic. The results of the literature research are primarily incorporated into the "State of the Art" chapter of the report.

In general, the research provides valuable insights for tackling the task at hand, and it can prevent the "reinvention of the wheel." In some cases, not only the current state of the art but also the fundamentals may need to be researched.

\underline{Frequently Used Sources for Literature Research}
\begin{itemize}
    \item Libraries
    \begin{itemize}
        \item TU Graz Library
        \item Libraries of other universities in Austria and abroad
        \item National libraries
        \item etc.
    \end{itemize}
    \item Patents
    \begin{itemize}
        \item Austrian Patent Office
        \item "Patent Guide" by PVA SH GmbH
        \item worldwide.espacenet.com
        \item depatisnet.dpma.de
        \item patft1.uspto.gov
        \item google.com/patents
    \end{itemize}
    \item Scientific Articles and Books
    \begin{itemize}
        \item sciencedirect.com
        \item scholar.google.com
        \item www.semanticscholar.org
        \item link.springer.com
        \item ieeexplore.ieee.org
    \end{itemize}    
\end{itemize}


%===============================================================================
\subsection{Structure of the Thesis}
\label{sec:structureOfThesis}

The following presents a possible structure for the thesis, with a detailed explanation of each section. It is emphasized that this is only a guideline and not a set of rigid rules.

\textbf{(a) Title Page} \\
A title page is included in the document template (available in the TeachCenter).

\textbf{(b) Abstract} \\
The abstract is independent of the text of the thesis. It should only contain keywords, numerical data, arguments, etc., that are included in the text of the thesis.

Recommended structure of the text:
\begin{enumerate}
    \item Paragraph: Brief introduction to the topic (starting point and problem statement)
    \item Paragraph: Method, implementation
    \item Paragraph: Results and conclusions
\end{enumerate}

The abstract should summarize the essential content of the thesis in as few words as possible (with precise wording). The abstract should not be overloaded with too many numerical values.

\underline{For Master's Theses:} It is advisable to use the same text for entry in TUGonline. The text should be a maximum of \num{4000} characters.

\textbf{(c) Abstract in English} \\
This is followed by the abstract in English. A free rendering of the content of the German abstract is better than a literal translation.

\textbf{(d) Table of Contents} \\
The table of contents should reproduce the structure of the thesis with decimal classification and page numbers, ideally up to three places for subchapters. After the main chapters, the following (without decimal classification) should follow:

\begin{itemize}
    \item Bibliography
    \item Appendices, annexes, etc.
    \item List of tables, list of figures, list of abbreviations (optional)
\end{itemize}

\textbf{(e) Main Body and Appendices} \\
Ensure a logical main structure (usually no more than 9 numbered main chapters), for example:

\begin{enumerate}
    \item Introduction
    \item Fundamentals
    \item State of the Art
    \item Methodology
    \item Results
    \item Summary/Conclusions
\end{enumerate}
\hspace{5mm}
Abbreviations (Nomenclature)

\vspace{1mm}
\hspace{5mm}
List of Figures

\vspace{1mm}
\hspace{5mm}
List of Tables

\vspace{1mm}
\hspace{5mm}
Bibliography

\vspace{5mm}
\underline{Introduction}

\begin{itemize}
    \item The introduction usually forms the first main chapter.
    \item It should introduce the topic and include the problem statement after a brief overview of the state of the art in the respective field.
    \item Long introductions to the topic should be avoided, especially when it concerns already frequently addressed topics.
    \item On the other hand, for less frequently addressed or 'new' topics, a longer introduction describing the principles and technical solutions, as well as the state of science and technology in this area, may be appropriate. In such cases, it should be considered whether the state of the art should be given its own main chapter.
    \item The introduction should also explain the motivation for the work.
    \item The conclusion of the introduction could, for example, be a brief preview of the upcoming chapters.
\end{itemize}

\underline{Main Chapters}

\begin{itemize}
    \item A main chapter on the technical/scientific fundamentals of the work may be useful.
    \item For a practical thesis, it is necessary to describe the methodology in a chapter.
    \item The chapter before the conclusion should be dedicated to the results of the work.
\end{itemize}

\underline{Summary/Conclusions}

\begin{itemize}
    \item In the conclusions, it is appropriate to first provide a brief overview of the problem statement and the method of execution.
    \item Then, a summary of the main results follows.
    \item The conclusion chapter represents the essence of the thesis and will most likely be read by outsiders after the abstract, which is why it must be particularly clear and precise.
\end{itemize}

\underline{Abbreviations (Nomenclature)}

\begin{itemize}
    \item An abbreviations list is useful when the same abbreviations/formula symbols are used repeatedly in different chapters.
    \item It is alphabetically ordered and contains ALL abbreviations and formula symbols used in the thesis.
\end{itemize}


%===============================================================================
\subsection{Data Storage and Backup}
\label{sec:dataStorageAndBackup}

If, for example, a large number of result files are expected in simulations, the storage format and the systematics of file naming should be clarified with the supervisor before starting the work. In general, filenames should be systematically named or organized so that important experimental/calculation parameters can already be recognized from the name. In addition, regular backups of the files are recommended, especially for the central text file. Check in advance before the planned printing date of the work (or periodically in between) whether the creation of a PDF file (which is generally used for printing) is possible without issues, particularly the quality of the images and the correct display of formulas.


%===============================================================================
\subsection{Style and Formal Criteria}
\label{sec:styleAndFormalCriteria}

In principle, it is assumed that scientific works do not need to be read and understood by completely non-expert individuals. However, it is advisable to make reading easier for the reader by fulfilling certain stylistic and formal criteria.


%===============================================================================
\subsubsection{General Form of the Thesis}
\label{sec:generalFormOfTheThesis}

When formatting the thesis, the author is generally given freedom regarding the form and appearance of the work. However, certain rules must be observed that are common and sensible when writing scientific papers.

\begin{itemize}
    \item Word processing preferably in MS Word or LaTeX. Templates for both are available in the Teach Center.
    \item The font size should be chosen so that the main text uses the font Times New Roman, size 11: Times New Roman size 11, Calibri size 12, Arial size 11, etc.
    \item Margins: Left at least 25 mm; other margins at least 20 mm (may be slightly exceeded on the right, for example, in tables or figures).
    \item Page numbers: Roman numerals up to and including the table of contents, Arabic numerals starting with "1" from the text of the thesis.
    \item Header with chapter number and chapter name.
    \item Use spell check, and if you are unsure about a spelling, resources like \url{http://www.duden.de/} for German-language works or \url{https://www.linguee.de/} for English-language works are helpful.
\end{itemize}


%===============================================================================
\subsubsection{Writing Style}
\label{sec:writingStyle}

\begin{itemize}
    \item Typically write in the present tense.
    \item Avoid using the 'I', 'We', or 'one' forms; instead, use passive constructions wherever possible.
    \item Use only words from the (German) language that you are familiar with. The use of foreign words can be nice, but it can also be embarrassing if used incorrectly.
    \item The highest priority when writing is readability and understandability! Complex sentences with numerous subordinate clauses are usually not appropriate.
    \item Through reading the technical literature, determine which technical terms, abbreviations, and symbols are commonly used in your field and prefer to use these. If unsure, ask your supervisor.
    \item Always use the same term when referring to a specific object, procedure, etc.
    \item Abbreviations should be spelled out at least once in the text and explained. Be sparing with abbreviations. If many abbreviations are used, include a list of abbreviations.
    \item Do not mix English and German abbreviations; instead, choose one language and be consistent.
    \item Avoid vague descriptions such as "the emissions are quite high" or "fairly high", etc.
    \item Define specific terms when they first appear.
    \item Overall, ensure a logical, systematic structure in your thesis.
\end{itemize}


%===============================================================================
\subsubsection{Chapters / Paragraphs}
\label{sec:chaptersParagraphs}

\begin{itemize}
    \item A main chapter should begin on a new page.
    \item Text should immediately follow a heading.
    \item At the beginning of a chapter, provide a brief overview of the content of the following sections. This should serve as a transition, not just a listing.
    \item Chapter numbering should be logical and sensible. The thesis should have a maximum of four heading levels (1.1.1.1) and no more than 9 main chapters, which is usually sufficient.
    \item Introduce subdivisions only when at least two elements at that level exist (3.1.1 is unnecessary if 3.1.2 does not exist).
    \item Paragraphs often make reading easier, but they should only be used where a shift in thought occurs.
    \item Paragraphs should be clearly recognizable: preferably with a blank line or increased line spacing.
    \item Use bullet points or other symbols of your choice for lists, with special emphasis using small Latin letters, small Roman numerals, or Arabic numerals.
    \item After completing all content corrections, check that no page breaks have occurred between tables or figures and their captions.
\end{itemize}


%===============================================================================
\subsubsection{Tables, Figures, Formulas, Cross-references}
\label{sec:tables_figures_formulas_cross_references}

\begin{itemize}
    \item All tables and figures should be numbered independently of each other, either continuously or chapter-wise.
    \item Periodically check whether cross-references are still valid.
    \item All tables and figures must be referred to at least once in the text.
    \item Tables and figures should, if possible, not exceed A4 size.
    \item If tables extend over multiple pages, indicate this on the second page.
    \item Avoid "relative references" (e.g., "in the following figure") and instead use "in Figure xy".
    \item Figures and tables should be self-explanatory, meaning the caption should make the essential information immediately apparent.
    \item In the text, describe in detail what is shown in a figure and what conclusions can be drawn from it. Do not assume the reader will interpret the figure correctly.
    \item Distinguish between what is actually (objectively) visible in the figure and what conclusions you have drawn from it.
    \item If figures or tables are taken from other publications, cite the source at the end of the legend or table heading, even if the citation already appears in the text.
    \item The font size in figures and tables should be roughly the same (possibly slightly smaller) than the main text. Experience shows that the font size in diagrams often ends up too small.
    \item Tables and figures should be placed as close as possible to their first citation in the text (so that page breaks are favorable), generally after the citation.
    \item Use logical line types (colors) and symbols when discussing parameter influences in figures.
    \item If two figures are presented side by side for comparison purposes (e.g., measurement results with vs. without insulation), they should have the same axis scaling and size.
    \item Equations should be numbered with numbers aligned to the right (right-aligned).
    \item When cross-referencing other chapters (or figures in other chapters), also provide a brief description of what information is found there, e.g., "As described in Section 3.1, an increased speed can...".
    \item Make at least one cross-reference to appendices in the main body of the text (e.g., "The measurement results in Appendix A...").
    \item Before submitting (either printed or as a PDF), check that all figures are of the correct quality and that formulas are displayed correctly. Issues with this often arise.
\end{itemize}


%===============================================================================
\subsubsection{Formal Writing Style}
\label{sec:formal_writing_style}

\begin{itemize}
    \item Decimal separator: use a comma (,) in German-language works and a period (.) in English-language works!
    \item Only write significant digits after the decimal point.
    \item Use fixed spaces (Word: Ctrl+Shift+Space; displayed as a formatting symbol: °, LaTeX: \raisebox{-0.8ex}{\~{}} (tilde)) before and after "=" and between numbers and units to prevent the unit from shifting to the next line or becoming stretched in justified text.
    \item Unit notation:
    \begin{itemize}
        \item In the text: without parentheses.
        \item In axis labels of diagrams: unit should not be in square brackets, but "in".
        \item In table headers: either after the label with "in", or preferably, introduce a separate row/column for units.
    \end{itemize}
    \item When multiplying variables in equations, use a centered small dot or a fixed space.
    \item Pay attention to correct, unambiguous parentheses usage.
    \item Use a hyphen for compound words and an en dash for pauses.
\end{itemize}