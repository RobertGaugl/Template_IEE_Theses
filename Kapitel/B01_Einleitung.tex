% !TEX root = ../Main.tex
%===============================================================================
%
% Einleitung.tex
%
% Autor: Robert Gaugl
%
% Inhalt: Lorem Ipsum
%
%===============================================================================

\section{Einleitung}

In diesem Kapitel soll erläutert werden, warum und wieso deine Fragestellung deiner Arbeit von Interesse ist. Außerdem können hier allgemeine Erklärungen rund um deine Arbeit beschrieben werden bzw. kann hier ein Überblick über die aktuellen Technologien gegeben werden. Ein ToDo kann mit dem Befehl \textbackslash todo\{text\} eingefügt \todo{Dies ist ein Beispiel ToDo.}werden.


\subsection{Unterkapitel}

So sieht ein Unterkapitel\footnote{In einer Fußnote können Details, welche nicht so wichtig sind um sie im Haupttext zu erklären, niedergeschrieben werden.} aus.

Um einen Zahlensalat ala {\glqq}1.2.1.3.4.5 Unterkapitel 5{\grqq} zu vermeiden, verwende wenn möglich maximal Überschriften mit drei (1.2.3 Überschrift) bis maximal vier (1.2.3.4 Überschrift) Zahlen in der Nummerierung. Im Inhaltsverzeichnis werden nur Überschriften mit drei Zahlen in der Nummerierung dargestellt.


\subsubsection{Bilder}

In Abbildung~\ref{fig:Baum} ist ein Beispiel für ein Bild mit entsprechender Bildunterschrift dargestellt. Jedes Bild muss im Text auch beschrieben und verwiesen werden. Bei nicht selbstgemachten Bildern die Quellenangabe nicht vergessen. Wenn ein Bild selber erstellt wurde, aber auf Zahlen einer anderen Quelle beruht, muss dies auch angegeben werden (Zahlen von [XY]).

\begin{figure}[H]
	\begin{center}
		\includegraphics{Abbildungen/Baum.png}
		\caption{Das ist ein schöner Baum}
		\label{fig:Baum}
	\end{center}
\end{figure}


\subsubsection{Tabellen}

Die Gestaltung einer Tabelle obligt dem Bacheloranden/Diplomanden. Sie sollten jedoch klar leserlich und wenn möglich ein einheitliches Design im ganzen Dokument besitzen.

Sollten Tabellen aus dem Internet verwendet werden, so sollten diese selbst neu geschrieben werden und mit einer entsprechenden Quelle versehen werden. Tabelle~\ref{tab:Tabelle1} ist ein Beispiel wie eine Tabelle aussehen kann.

Bei uns am Institut gibt es keine einheitliche Regel ob die Tabellenbeschriftung oberhalb oder unterhalb der Tabelle zu stehen hat. Wie bei Abbildungen gilt allerdings: Jede Tabelle muss im Text beschrieben und verwiesen werden.

\begin{table}[H]
	\centering 
	\begin{tabular}{c c c c c}
		\hline
		\textbf{Szenarien}		& \textbf{RAV } 		& \textbf{Mehrkosten}	& \textbf{Mehrkosten}	& \textbf{Mehrkosten}	\\	
		& [\%]       			& [\euro/MWh]  			& [Mio.\euro] 			& [\euro/MWh] 			\\
		\hline
		\rowcolor[gray]{0.9}
		Szenario 1				& 0,5	 				& 5						& 0,242					& 14,9					\\
		Szenario 2				& 0,5	  				& 10					& 0,390					& 24,1					\\
		\rowcolor[gray]{0.9}
		Szenario 3				& 0,5   				& 20					& 0,692					& 42,8					\\
		Szenario 4				& 1	  					& 5						& 0,331					& 20,5					\\
		\hline \hline
	\end{tabular}
	\caption{Dies ist eine Testtabelle}\label{tab:Tabelle1}
\end{table}


\subsubsection{Formeln}

Anders als bei Abbildungen und Tabellen ist die Formelnummerierung auf der rechten Seite zu platzieren. Anders als Microsoft Word erledigt dies LaTeX allerdings automatisch.

Das Zeichen für eine Multiplikation wird durch Eingabe von '\textbackslash cdot' innerhalb der Formelumgebung erreicht. Das fälschlicherweise gern verwendete '*'-Zeichen steht für eine Faltung und nicht für eine Multiplikation!

Formel~\ref{eq:Ohmsches_Gesetz} zeigt ein Beispiel für eine korrekte Formel mit entsprechender Formelnummerierung und Erklärung der Variablen.

\begin{align}
	\label{eq:Ohmsches_Gesetz}
	U &= R \cdot I
\end{align}
\vspace*{-1cm}
\begin{table}[H]
	\begin{tabular}{@{}p{1cm}@{}p{1cm}<{\dotfill}@{}p{\dimexpr\linewidth-5cm}}
		& $U$ & Spannung in $V$  \\
		& $R$ & Widerstand in $\Omega$ \\
		& $I$ & Strom in $A$ 
	\end{tabular}
\end{table}

Die Spannung $U$ berechnet sich nach dem ohm'schen Gesetz aus der Multiplikation aus Strom $I$ und Widerstand $R$.


\subsubsection{Zahlen und Einheiten}

Die Formatierung von Zahlen muss in der ganzen Arbeit einheitlich sein. 
Üblicherweise wird in deutschen Arbeiten ein Punkt als Tausendertrennzeichen verwendet (1.000~kV), seltener auch ein gebundenes Leerzeichen (1~000~kV) welches durch eine Tilde '$\sim$' im LaTeX-Code erreicht wird. Ein gebundenes Leerzeichen verhindert, dass die Zahl aufgeteilt wird und in zwei unterschiedlichen Zeilen steht.

Als Dezimaltrennzeichen wird ein Komma verwendet (13,76~m).

Wenn Werte aus englischsprachigen Quellen kopiert werden, muss darauf geachtet werden, dass die richtigen Tausender- und Dezimaltrennzeichen verwendet werden, da im englischen Kommas als Tausendertrennzeichen und Punkte als Dezimaltrennzeichen verwendet werden. Wird die englische und deutsche Formatierung vermischt, weiß man nicht mehr ob mit 8,763 nun {\glqq}Acht Komma Sieben Sechs Drei{\grqq} oder {\glqq}Achttausendsiebenhundertdreiundsechzig{\grqq} gemeint ist.

Wird die Arbeit in Englisch verfasst, so wird als Dezimaltrennzeichen ein Punkt und als Tausendertrennzeichen bevorzugt ein gebundenes Leerzeichen (Tilde  '$\sim$') oder ein Komma verwendet.

Zwischen der Zahl und der Einheit ist wiederum ein gebundenes Leerzeichen (Tilde '$\sim$') einzufügen (10~kV), damit Zahl und Einheit nicht durch einen Zeilenumbruch getrennt werden. Außnahmen sind Prozent und Gradangaben: Diese erfolgen ohne Leerzeichen zwischen Zahl und Einheit (50\%, 10$^\circ$C).