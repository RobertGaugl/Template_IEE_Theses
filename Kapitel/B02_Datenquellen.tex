% !TEX root = ../Main.tex
%===============================================================================
%
% Datenquellen.tex
%
% Autor: Stigler, Süßenbacher
%
% Inhalt: Lorem Ipsum
%
%===============================================================================

\section{Datenquellen}
\label{sec:datenquellen}

Daten sind logisch gruppierte Informationseinheiten, welche in einem Bedeutungskontext interpretiert werden müssen, um Informationen gewinnen zu können.

\subsection{Datenqualität}
\label{sec:datenqualität}

Die Datenqualität beschreibt die Korrektheit und Relevanz von Informationen, gibt an wie gut die Daten zur Beschreibung der Realität geeignet sind und gibt Auskungt über die Verlässlichkeit und Planungssicherheit der Informationen.

\underline{4 Dimensionen in der Datenqualität}
    \begin{enumerate}
        \item \textbf{Informationszugang:} Daten einfach und auf direktem Weg zugänglich   
        \item \textbf{Darstellung:} Daten sind verständlich, glaubwürdig (Zertifikaten, Einhaltung von Qualitätsstandards, hoher Aufwand), eindeutig auslegbar integer
        \item \textbf{Informationszusammenhang:} Daten sind relevant, erbringen einen Zusatznutzen, sind aktuell, vollständig und in einem aussagekräftigen Umfang vorhanden
        \item \textbf{Eigenwert:} Daten sind fehlerfrei, objektiv, glaubwürdig und von einer Informationsquelle hoher Vertrauenswürdigkeit und Kompetenz
    \end{enumerate}

\subsection{Datenklassifikation}
\label{sec: datenklassifikation}

Daten können unterteilt werden in
\begin{itemize}
    \item Zeitbehaftete Daten (Zeitreihen, Paneldaten, Trenddaten)
    \item Globale Daten
    \item Sektorale Daten
    \item Absolute- und Relative Daten
\end{itemize}

\newpage

\subsection{Allgemeine Datenquellen}
\label{sec: allgemeine datenquellen}

\underline{Weltweit}
\begin{itemize}
    \item \textbf{Weltbank:} www.worldbank.org
    \item \textbf{OECD:} www.oecd.org
    \item \textbf{UNO:} www.un.org
    \item \textbf{Internationaler Währungsfond:} www.imf.org
    \item \textbf{Global Statistics:} geohive.ie
    %\item Datenquellenübersicht auf MyGeo
\end{itemize}

\underline{Europa}
\begin{itemize}
    \item \textbf{EUROSTAT} ec.europa.eu/eurostat
    \item \textbf{EU-Kommission:} ec.europa.eu
    \item \textbf{Nationale Statistische Zentralämter und Ministerien}
\end{itemize}

\underline{Österreich}
\begin{itemize}
    \item \textbf{Statistik Austria:} www.statistik.at
    \item \textbf{Umweltbundesamt:} www.umweltbundesamt.at
    \item \textbf{Ministerien} (Lebensministerium, BMWA, ...)
    \item \textbf{Wirtschaftskammer:} www.wko.at
    \item \textbf{Statistikbüro der Bundesländer:} www.verwaltung.steiermark.at
\end{itemize}

\newpage

\subsection{Datenquellen Energiebereich}
\label{datenquellen energiebereich - weltweit}

\underline{Internationale und globle Energiestatistiken}
\begin{itemize}
    \item \textbf{Department of Energy (US-Ministerium]:} www.energy.gov
    \item \textbf{International Energy Agency- IEA:} www.iea.org
    \item \textbf{British Petrolium - BP:} www.deutschebp.de
\end{itemize}

\subsection{Datenquellen EI.-Wirtschaft}
\label{sec: datenquellen ei.-wirtschaft}

\underline{Welt}
\begin{itemize}
    \item \textbf{World Energy Counsil -WEC:} www.worldenergy.org
    \item \textbf{Global Energy Network Institute - GENI:} www.geni.org
\end{itemize}

\underline{Europa}
\begin{itemize}
    \item \textbf{Union for the Co-ordination of Transmission of Electricity - UCTE:} www.ucte.org
    \item \textbf{European Transmission System Operators - ENTSO-E:} www.entsoe.eu
    \item \textbf{Eurelectric:} www.eurelectric.org
\end{itemize}

\underline{Österreich}
\begin{itemize}
    \item \textbf{E-Control:} www.e-control.at
\end{itemize}

\newpage

\subsection{Kostenpflichtige Datendienste}
\label{sec: kostenpflichtige datendienste}

\underline{Kraftwerksdatenbank}
\begin{itemize}
    \item \textbf{Platts:} www.platts.com
\end{itemize}

\underline{Brennstoffpreise}
\begin{itemize}
    \item Platts, Bloomberg, ...
\end{itemize}

\underline{Strompreise}
\begin{itemize}
    \item Platts, ...
    \item EXAA, EEX, andere Börsen, ...
\end{itemize}

\underline{Countyreports}
\begin{itemize}
    \item \textbf{Enerdata:} www.enerdata.fr
    \item Namhafte Unternehmensberater (Pöyry, Bosten Consulting, ...)
  
\end{itemize}



\subsection{Geographische / Meteorologische Datenquellen}
\label{geographische / meteorologische datenquellen}

\underline{www.wetter-online.de}
\begin{itemize}
    \item Pegelstände für Deutschland
\end{itemize}

\underline{Global Runoff Data Centre}
\begin{itemize}
    \item Weltdatenzentrum Abfluss, Weltdatenbank mit internationalen Abflussdaten
    \item GRDC
\end{itemize}

\underline{Nationale Hydrographische Dienste}
\begin{itemize}
    \item z.B. Serbien
\end{itemize}
