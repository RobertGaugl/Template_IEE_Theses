% !TEX root = ../Main.tex
%===============================================================================
%
% Quellenangaben.tex
%
% Autor: Robert Gaugl
%
% Inhalt: Lorem Ipsum
%
%===============================================================================

\section{Quellenangaben}

Unter jeder Überschrift hat zumindest ein kleiner Text zu stehen. Es gibt keine zwei aufeinanderfolgenden Überschriften ohne Text dazwischen.


\subsection{Beispiele für Quellenangaben}

Viele LaTeX-Editoren haben eigene Funktionen zum Erstellen von Quellenangaben mit BibTeX oder BibLaTeX. Als Zitierstandard wird der IEEE-Zitierstil empfohlen, welcher voreingestellt ist. Sollte ein anderer Zitierstil erwünscht sein, so kann dies in der Main.tex Datei geändert werden.

In TeXstudio können Quellenangaben durch den Menüeintrag '\textit{Bibliographie -> Literatureintrag einfügen}' hinzugefügt werden. Dadurch wird der Literatureintrag in Literatur.bib\footnote{Diese Datei kann natürlich mit TeXstudio geöffnet werden um sie anzupassen.} hinzugefügt. 

Dies ist eine Beispiel für einen Quellenverweis~\cite{Hunt2019}.

Verwendet man zur Literaturverwaltung Mendeley oder Cite (was vor allem für umfangreichere Arbeiten wie Masterarbeiten dringendst empfohlen wird), so gibt es die Möglichkeit die in den jeweiligen Programmen ausgewählten Literatureinträge als .bib-Datei zu exportieren. Dann entfällt die oben beschriebene händische Eingabe der Literatureinträge über den Menüeintrag '\textit{Bibliographie -> Literatureintrag einfügen}'.

Bei Zitaten wird zwischen direkten und indirekten (sinngemäßen) Zitaten unterschieden.


\subsubsection{Direktes Zitat}

Ein direktes Zitat übernimmt die Aussage wortwörtlich. Diese sollten in Anführungszeichen stehen und kursiv geschrieben werden. Diese sind nur in wenigen Ausnahmefällen erlaubt.

Beispiel für ein wortwörtlich übernommenes Zitat:

\begin{center}
		{\glqq}\textit{Zwei Dinge sind unendlich, das Universum und die menschliche Dummheit, aber bei dem Universum bin ich mir noch nicht ganz sicher.}{\grqq} \cite{Einstein1941}
\end{center}


\subsubsection{indirektes Zitat}

Standardmäßig sind indirekte Zitate in einer wissenschaftlichen Arbeit zu verwenden. Das heißt, man liest sich die Quellen durch und gibt danach den Inhalt in \textbf{eigenen Worten} wieder.

Auch bei nicht wortwörtlich übernommenen Quellen muss ein Quellenverweis hinzugefügt werden, damit nachvollzogen werden kann woher der Autor das Wissen hat. 

Bezieht sich die Quellenangabe nur auf den \textbf{vorangegangenen Satz}, so befindet sich die Quellenangabe \textbf{vor dem Satzzeichen}. Beispiel:

Die Europäische Kommission hat folgende Ziele für 2020 definiert: 20\% weniger Treibhausgase, 20\% Erneuerbare Energien und 20\% bessere Energieeffizienz~\cite{EC2008}.

Ist jedoch ein \textbf{ganzer Absatz} von einer Quelle, so befindet sich die Quellenangabe \textbf{nach dem Satzzeichen des letzten Satzes} im Absatz. Ein Beispiel dafür wäre folgender Absatz:

In diesem Kapitel wird beschrieben, wie die Umwandlung von Licht bzw. solarer Strahlung in elektrische Energie innerhalb einer Solarzelle erfolgt. Es wird der Unterschied zwischen dem äußeren und dem inneren Photoeffekt erläutert und beschrieben, wieso nur der innere Photoeffekt bei einer Solarzelle von Bedeutung ist. Außerdem wird auch der photovoltaische Effekt erklärt. Sowohl der innere Photoeffekt, als auch der photovoltaische Effekt sind für die grundsätzliche Beschreibung der Funktionsweise erforderlich. Damit Strom fließen kann, benötigt man zuerst freie Ladungsträger und eben diese werden durch den inneren Photoeffekt erzeugt. Damit diese freien Elektronen aber nicht sofort wieder mit einem Loch rekombinieren, wird der photovoltaische Effekt eines pn-Übergangs ausgenutzt.~\cite{Finke2012}

\textbf{Jeder Absatz} muss mit einer \textbf{Quellenangabe} belegt werden, sofern es sich um die Wiedergabe von Inhalten einer anderen Quelle handelt. Auch wenn\textbf{ mehrere Absätze} von der \textbf{gleichen Quelle} stammen, muss \textbf{jeder Absatz} mit einer \textbf{Quellenangabe} ausgestattet sein.\

Basiert ein \textbf{ganzes Kapitel} auf der \textbf{gleichen Quelle} (sollte nur in Ausnahmefällen gemacht werden), so kann dies auch folgendermaßen angegeben werden:

Sofern nicht anders angegeben, stützt sich das folgende Kapitel auf das Buch {\glqq}Netzgekoppelte Photovoltaikanlagen{\grqq} von Jürgen Schlabbach~\cite{Schlabbach2011}.